\documentclass{article}

\usepackage{arxiv}

\usepackage[utf8]{inputenc} % allow utf-8 input
\usepackage[T1]{fontenc}    % use 8-bit T1 fonts
\usepackage{lmodern}        % https://github.com/rstudio/rticles/issues/343
\usepackage{hyperref}       % hyperlinks
\usepackage{url}            % simple URL typesetting
\usepackage{booktabs}       % professional-quality tables
\usepackage{amsfonts}       % blackboard math symbols
\usepackage{nicefrac}       % compact symbols for 1/2, etc.
\usepackage{microtype}      % microtypography
\usepackage{graphicx}

\title{A template for the \emph{arxiv} style}

\author{
    Randy
    \thanks{Use footnote for providing further information about author}
   \\
    Department of Biostatistics and Bioinformatics \\
    University of Colorado \\
  Aurora, CO, USA \\
  \texttt{\href{mailto:xin.2.jin@cuanschutz.edu}{\nolinkurl{xin.2.jin@cuanschutz.edu}}} \\
  }

% Pandoc syntax highlighting
\usepackage{color}
\usepackage{fancyvrb}
\newcommand{\VerbBar}{|}
\newcommand{\VERB}{\Verb[commandchars=\\\{\}]}
\DefineVerbatimEnvironment{Highlighting}{Verbatim}{commandchars=\\\{\}}
% Add ',fontsize=\small' for more characters per line
\usepackage{framed}
\definecolor{shadecolor}{RGB}{248,248,248}
\newenvironment{Shaded}{\begin{snugshade}}{\end{snugshade}}
\newcommand{\AlertTok}[1]{\textcolor[rgb]{0.94,0.16,0.16}{#1}}
\newcommand{\AnnotationTok}[1]{\textcolor[rgb]{0.56,0.35,0.01}{\textbf{\textit{#1}}}}
\newcommand{\AttributeTok}[1]{\textcolor[rgb]{0.77,0.63,0.00}{#1}}
\newcommand{\BaseNTok}[1]{\textcolor[rgb]{0.00,0.00,0.81}{#1}}
\newcommand{\BuiltInTok}[1]{#1}
\newcommand{\CharTok}[1]{\textcolor[rgb]{0.31,0.60,0.02}{#1}}
\newcommand{\CommentTok}[1]{\textcolor[rgb]{0.56,0.35,0.01}{\textit{#1}}}
\newcommand{\CommentVarTok}[1]{\textcolor[rgb]{0.56,0.35,0.01}{\textbf{\textit{#1}}}}
\newcommand{\ConstantTok}[1]{\textcolor[rgb]{0.00,0.00,0.00}{#1}}
\newcommand{\ControlFlowTok}[1]{\textcolor[rgb]{0.13,0.29,0.53}{\textbf{#1}}}
\newcommand{\DataTypeTok}[1]{\textcolor[rgb]{0.13,0.29,0.53}{#1}}
\newcommand{\DecValTok}[1]{\textcolor[rgb]{0.00,0.00,0.81}{#1}}
\newcommand{\DocumentationTok}[1]{\textcolor[rgb]{0.56,0.35,0.01}{\textbf{\textit{#1}}}}
\newcommand{\ErrorTok}[1]{\textcolor[rgb]{0.64,0.00,0.00}{\textbf{#1}}}
\newcommand{\ExtensionTok}[1]{#1}
\newcommand{\FloatTok}[1]{\textcolor[rgb]{0.00,0.00,0.81}{#1}}
\newcommand{\FunctionTok}[1]{\textcolor[rgb]{0.00,0.00,0.00}{#1}}
\newcommand{\ImportTok}[1]{#1}
\newcommand{\InformationTok}[1]{\textcolor[rgb]{0.56,0.35,0.01}{\textbf{\textit{#1}}}}
\newcommand{\KeywordTok}[1]{\textcolor[rgb]{0.13,0.29,0.53}{\textbf{#1}}}
\newcommand{\NormalTok}[1]{#1}
\newcommand{\OperatorTok}[1]{\textcolor[rgb]{0.81,0.36,0.00}{\textbf{#1}}}
\newcommand{\OtherTok}[1]{\textcolor[rgb]{0.56,0.35,0.01}{#1}}
\newcommand{\PreprocessorTok}[1]{\textcolor[rgb]{0.56,0.35,0.01}{\textit{#1}}}
\newcommand{\RegionMarkerTok}[1]{#1}
\newcommand{\SpecialCharTok}[1]{\textcolor[rgb]{0.00,0.00,0.00}{#1}}
\newcommand{\SpecialStringTok}[1]{\textcolor[rgb]{0.31,0.60,0.02}{#1}}
\newcommand{\StringTok}[1]{\textcolor[rgb]{0.31,0.60,0.02}{#1}}
\newcommand{\VariableTok}[1]{\textcolor[rgb]{0.00,0.00,0.00}{#1}}
\newcommand{\VerbatimStringTok}[1]{\textcolor[rgb]{0.31,0.60,0.02}{#1}}
\newcommand{\WarningTok}[1]{\textcolor[rgb]{0.56,0.35,0.01}{\textbf{\textit{#1}}}}

% tightlist command for lists without linebreak
\providecommand{\tightlist}{%
  \setlength{\itemsep}{0pt}\setlength{\parskip}{0pt}}

% From pandoc table feature
\usepackage{longtable,booktabs,array}
\usepackage{calc} % for calculating minipage widths
% Correct order of tables after \paragraph or \subparagraph
\usepackage{etoolbox}
\makeatletter
\patchcmd\longtable{\par}{\if@noskipsec\mbox{}\fi\par}{}{}
\makeatother
% Allow footnotes in longtable head/foot
\IfFileExists{footnotehyper.sty}{\usepackage{footnotehyper}}{\usepackage{footnote}}
\makesavenoteenv{longtable}

% Pandoc citation processing
\newlength{\cslhangindent}
\setlength{\cslhangindent}{1.5em}
\newlength{\csllabelwidth}
\setlength{\csllabelwidth}{3em}
\newlength{\cslentryspacingunit} % times entry-spacing
\setlength{\cslentryspacingunit}{\parskip}
% for Pandoc 2.8 to 2.10.1
\newenvironment{cslreferences}%
  {}%
  {\par}
% For Pandoc 2.11+
\newenvironment{CSLReferences}[2] % #1 hanging-ident, #2 entry spacing
 {% don't indent paragraphs
  \setlength{\parindent}{0pt}
  % turn on hanging indent if param 1 is 1
  \ifodd #1
  \let\oldpar\par
  \def\par{\hangindent=\cslhangindent\oldpar}
  \fi
  % set entry spacing
  \setlength{\parskip}{#2\cslentryspacingunit}
 }%
 {}
\usepackage{calc}
\newcommand{\CSLBlock}[1]{#1\hfill\break}
\newcommand{\CSLLeftMargin}[1]{\parbox[t]{\csllabelwidth}{#1}}
\newcommand{\CSLRightInline}[1]{\parbox[t]{\linewidth - \csllabelwidth}{#1}\break}
\newcommand{\CSLIndent}[1]{\hspace{\cslhangindent}#1}

\begin{document}
\maketitle


\begin{abstract}
The traditional predictive modeling approach mainly relies on global
inference and universal modelling with all available data. However, such
approach may overlook the cultural diversity and genetic heterogeneity
for patients. using fewer but more similar data could get higher
predictive performance than using overall available data. Recently,
curving matching prediction methods, based on predictive modeling with
similar matching donars, has been successfully applied in the medical
data analysis. This curve matching prediction aquires the information of
``people-like-me'', with the nearest neighbors of predictive mean. The
predictive mean matching can avoid the dataset noise and model
mispecification, with respect to given metrics. Through exhaustive
comparisons for specific target and the most similar matching
donor-cohort with predictive mean, an assessment specifc to the given
patient can help in identifying his similar patients. In this paper,

The model is implicit (Little and Rubin 2002), which means that there is
no need to define an explicit model for the distribution of the missing
values. Because of this, predictive mean matching is less vulnerable

In contrast, the imputations created by predictive mean matching follow
the data quite nicely, even though the predictive mean itself is clearly
off-target.

There are different strategies for defining the set and number of
candidate donors. Predictive mean matching performs very badly when
\(d\) is small and there are lots of ties for the predictors among the
individuals to be imputed.

The reason is that the tied individuals all get the same imputed value
in each imputed dataset when \(d = 1\) (Ian White, personal
communication). Setting \(d\) to a high value (say \(n/10\)) alleviates
the duplication problem, but may introduce bias since the likelihood of
bad matches increases.

Schenker and Taylor (1996), evaluated \(d = 3\), \(d = 10\) and an
adaptive scheme. The adaptive method was slightly better than using a
fixed number of candidates, but the differences were small. compared
various settings for \(d\), and found that \(d = 5\) and \(d = 10\)
generally provided the best results.
\end{abstract}

\keywords{
    predictive mean matching
   \and
    personalized data-driven prediction
   \and
    multiple imputation
   \and
    Mahalanobis distance
   \and
    people-like-me
  }

\hypertarget{introduction}{%
\section{Introduction}\label{introduction}}

Disease populations, clinical practice, and healthcare systems are
constantly evolving. This can result in clinical prediction models
quickly becoming outdated and less accurate over time. A potential
solution is to develop `dynamic' prediction models capable of retaining
accuracy by evolving over time in response to observed changes.
\href{https://diagnprognres.biomedcentral.com/articles/10.1186/s41512-018-0045-2}{1}

Most commonly, through exhaustive comparisons between a given patient
and a cohort of existing patients, an assessment specifc to the given
patient can help in identifying his similar patients. The purpose of
``people-like-me'' methods is to find the donors Most commonly, through
exhaustive comparisons between a given patient and a cohort of existing
patients, an assessment specifc to the given patient can help in
identifying his similar patients.

s who agreed the most with each patient. Te result suggested that using
fewer but more similar data could get higher predictive performance than
using overall available data. David et al.~{[}20{]} proposed an
algorithm for the anomaly detection and characterization on the basis of
the Euclidean distance

The results demonstrated that personalized predictive models showed a
higher performance. Many previous studies usually calculated the patient
similarity using single similarity measures (e.g., Euclidean distance,
cosine distance, and Mahalanobis distance), and most of them did not
take the importance of patient features into consideration while
calculating the similarity. In this study, we aimed to investigate in
depth the patient similarity in the following two aspects. One is using
diferent similarity metrics for diferent types of feature data. Te other
is assigning diferent weights (importance) to patient features when
integrating feature similarities into a patient similarity

many studies have found secondary use such as patient trajectory
modeling, disease inference and clinical decision support system. It is
recommended to denoise data before building a global predictive model,
which will be time consuming and challenging to represent and model. In
this context, individualized predictive modeling based on patient
similarity emerged and was shown to be adjustable for individual
patients. Employing patient similarity helps to identify a precision
cohort for an index patient, which will then be used to train a
personalized model. Accordingly, when building a predictive model for an
index patient, training samples are determined as ``patients like me,''
instead of using all available training samples in a conventional way.
``Patients like me'' are selected from the training sample set on the
basis of similarity between the index patient and each training sample.
Of note, based on patient similarity, patients with noisy data are less
likely to be selected as similar patients of an index patient for the
reason of the less similarity between them. Patient similarity is
usually measured by considering information on demographics, disease
history, comorbidities, laboratory tests, hospitalizations, treatment,
and pharmacotherapy. Such data are easily extracted from the EMR for
tens of millions of patients {[}13{]}. In this study, we defned a
patient as a vector in a d-dimensional feature space. Ten, a
multi-dimensional approach to estimate patient similarity was proposed.
To demonstrate the efectiveness of the proposed similarity measure, the
most similar patients were retrieved to build personalized models to
predict the diabetes status of a given patient

Four methods has been proposed by Andridge and Little(2010) Hot deck
imputation:

\begin{enumerate}
\def\labelenumi{\arabic{enumi}.}
\item
  ``The chosen threshold'' Choose a threshold, and take all i for which
  as candidate donors for imputing j. Randomly sample one donor from the
  candidates, and take its yi as replacement value.
\item
  ``The nearest neighbor'' , i.e., the case i for which \textbar yi yj
  \textbar{} is minimal as the donor. This is known as ``nearest
  neighbor hot deck,'' ``deterministic hot deck'' or ``closest
  predictor.''
\item
  ``Sampling from k neighbors'' Find the d candidates for which
  \textbar yi yj \textbar{} is minimal, and sample one of them. Usual
  values for d are 3, 5 and 10. There is also an adaptive method to
  specify the number of donors (Schenker and Taylor, 1996).
\item
  ``Sampling from probability'' Sample one donor with a probability that
  depends on \textbar yi yj \textbar{} (Siddique and Belin, 2008).
\end{enumerate}

Existing approaches to modeling with predictive mean matching mainly
rely on single predictive mean value on one single time point and a
fixed number of candidate donors. These simplified modeling strategies
may give a rise of several problems.

The GAMLSS method (Rigby and Stasinopoulos, 2005; Stasinopoulos et al.,
2017) extends both the generalized linear model and the generalized
additive model. A unique feature of GAMLSS is its ability to specify a
(pos- sibly nonlinear) model for each of the parameters of the
distribution, thus giving rise to an extremely flexible toolbox that can
be used to model almost any distribution. The gamlss package contains
over 60 built-in distributions. Each distribution comes with a function
to draw random variates, so once the gamlss model is fitted, it can also
be used to draw imputations.

Various metrics are possible to define the distance between the cases.
The predictive mean matching metric was proposed by Rubin (1986) and
Little (1988). This metric is particularly useful for missing data
applications because it is optimized for each target variable
separately.

The predicted value only needs to be a convenient one-number summary of
the important information that relates the covariates to the target.
Calculation is straightforward, and it is easy to include nominal and
ordinal variables.

The work presented here expands our previous preliminary study, as we 1)
further assessed the adequacy of other existing embeddings for modeling
medical concept dependence, 2) leveraged the similarity model by
considering each diagnosis of ciliopathy as index (as opposed to using
average similarity with all diagnosed patients) to take into account the
high heterogeneity of ciliopathies, and 3) applied the developed model
to two large-scale unbalanced datasets containing approximately 10,000
and 60,000 controls with kidney manifestations in the clinical data
warehouse

\begin{enumerate}
\def\labelenumi{\arabic{enumi}.}
\setcounter{enumi}{7}
\tightlist
\item
  Ng K, Sun J, Hu J, Wang F. Personalized predictive modeling and risk
  factor identifcation using patient similarity. AMIA Summits Transl Sci
  Proc. 2015;2015:132--6.
\item
  Whellan DJ, Ousdigian KT, Alkhatib SM, Pu W, Sarkar S, Porter CB,
  Pavri BB, O'Connor CM, Investigators PS. Combined heart failure device
  diagnostics identify patients at higher risk of subsequent heart
  failure hospitalizations: results from PARTNERS HF (program to access
  and review trending information and evaluate correlation to symptoms
  in patients with heart failure) study. J Am Coll Cardiol.
  2010;55(17):1803--10.
\item
  Sepanski RJ, Godambe SA, Mangum CD, Bovat CS, Zaritsky AL, Shah SH.
  Designing a pediatric severe sepsis screening tool. Front Pediatr.
  2014;2(56):56.
\item
  Wu J, Roy J, Stewart W. Prediction modeling using EHR data:
  challenges, strategies, and a comparison of machine learning
  approaches. Med Care. 2010;48(6 Suppl):S106.
\end{enumerate}

There are different strategies for defining the set and number of
candidate donors. Setting d = 1 is generally considered to be too low,
as it may reselect the same donor over and over again. Predictive mean
matching performs very badly when d is small and there are lots of ties
for the predictors among the individuals to be imputed. The reason is
that the tied individuals all get the same imputed value in each imputed
dataset when d = 1 (Ian White, personal communication). Setting d to a
high value (say n/10) alleviates the duplication problem, but may
introduce bias since the likelihood of bad matches increases.

Schenker and Taylor (1996) evaluated d = 3, d = 10 Morris et al.~(2014)
compared various Kleinke (2017) found that d = 5 may be too high for
sample size lower than n = 100, and suggested setting d = 1 for better
point estimates for small samples. Gaffert et al.~(2016) explored
scenarios in which candidate donors have different probabilities to be
drawn, where the probability depends on the distance between the donor
and recipient cases.

Instead a closeness parameter needs to be specified, and this was made
adaptive to the data. An advantage of using all donors is that the
variance of the imputations can be corrected by the Parzen correction,
which alleviates concerns about insuffi- cient variability of the
imputes. Their simulations showed that with a small sample (n = 10), the
adaptive method is clearly superior to methods with a fixed donor pool.

an adaptive method for setting d could improve small sample behavior.
Meanwhile, the number of donors can be changed through the donors
argument.

Because of the bias, the coverage is lower than nominal. For missing x
the bias is much smaller. Setting d to a lower value, as recommended by
Kleinke (2017), improves point estimates, but the magnitude of the
effect depends on whether the missing values occur in x or y. For the
sample size n = 1000 predictive mean matching appears well calibrated
for d = 5 for missing data in y, and has slight undercoverage for
missing data in x. Note that Table 3.3 in the first edition of this book
presented incorrect information because it had erroneously imputed the
data by norm instead of pmm.

The traditional method does not work for a small number of predictors.
Heitjan and Little (1991) report that for just two predictors the
results were ``disastrous.'' The cause of the problem appears to be
related to their use

The method is robust against misspecification of the imputation model,
yet performs as well as theoretically superior methods. In the context
of missing covariate data, Marshall et al.~(2010a) concluded that
predictive mean matching ``produced the least biased estimates and
better model per- formance measures.''

The method works best with large samples, and provides imputations that
possess many characteristics of the complete data. Predictive mean
matching cannot be used to extrapolate beyond the range of the data, or
to interpolate within the range of the data if the data at the interior
are sparse. Also, it may not perform well with small datasets. Bearing
these points in mind, predictive mean matching is a great all-around
method with exceptional properties.

there are two reasons to move beyond the predictive distance used in PMM
and investigate an alternative metric. Firstly, PMM requires users of
curve matching to select a particular future time point to base the
matches on (e.g.~14 months of age). In some cases, it may be difficult
to choose this time point, especially when the `future' is more vaguely
defined as a time interval.{[}4{]} Secondly, the predictive distance may
make the matches look unconvincing. The trajectories of the selected
donors may all be close to the prediction for the target child at 14
months, but this does not imply that the histories are identical. After
all, different profiles may lead to the same predicted value.

Consequently, the curves of some of the matches may be quite far from
the curve of the target child. Some users of curve matching feel that
such discrepancies are undesirable, as these matches do not appear to be
people-like-me.{[}4{]} It is useful to investigate these shortcomings
not only for improving growth prediction but also for other applications
of multiple imputation, such as patient recovery after an operation,
prediction of longevity, and decision-making when more than one
treatment is available. {[}3{]}

\hypertarget{methods}{%
\section{methods}\label{methods}}

Therefore, the information of these children at a later age is
available. The first step is to fit a linear regression model on the
donor database. Then, this model is used to predict the values for all
donors and for the target at a certain point in the future, for example
at 14 months. Finally, the distance between the predicted value of each
of the donors and the predicted value of the target is calculated, which
is referred to as the predictive distance. A number of donors -- usually
five - with the smallest predictive distance are selected as the best
matches. Their growth curves are then plotted and point estimates can be
calculated by averaging the measurements. The growth patterns of the
matched children thus suggest how the target child might develop in the
future.

the practical implementation and use of curve matching can in theory be
improved by combining the predictive distance with another distance
measure, thus creating a ``blended distance'' measure. Such a blended
metric would take into account historical similarity between the donors
and the target. For example, when blending the predictive distance with
the Mahalanobis distance, more weight is given to similarities between
units in the full predictor space. This would theoretically lead to the
selection of donors with profiles more similar to the target, and
therefore to the selection of true people-like-me. The objective of this
study is to implement such a blended distance measure and to investigate
its properties, blend ratio, and the validity of its resulting
inferences.

So instead of using one predictive distance at particular future time
point. we use multiple time; The PD is the distance between the
predicted value of a donor and the predicted value of the target at a
particular future time point. The MD is defined as the distance between
two N dimensional points scaled by the variation in each component of
the point.

\hypertarget{headings-first-level}{%
\section{Headings: first level}\label{headings-first-level}}

as the profiles of the matched donors can substantially differ from the
profile of the target. similarity between the curves of the donors and
the target can be taken into account by combining the predictive
distance with the Mahalanobis distance.

. The results show that blending towards the Mahalanobis distance leads
to worse performance in terms of bias, coverage, and predictive power.
Simulation study II evaluates the blended metric in a setting where a
single value is imputed. The results show that a property of blending is
the bias-variance trade off. Giving more weight to the Mahalanobis
distance leads to less variance in the imputations, but less accuracy as
well.

We used the following steps to train and test both PLM and LMM
prediction approaches: (1) build the approach using the training data,
(2) examine prediction performance and tune the approach using the
training data (i.e., within-sample testing), and (3) test the accuracy
and precision of each approach using the testing data (i.e.,
out-of-sample testing). We compared performance of PLM and LMM
predictions in terms of accuracy and precision across all individuals
and all timepoints.

When no more than 30\% of the whole training sample (i.e., 3000 samples)
were used to build the models, all three personalized predictive models
outperformed the corresponding traditional models, which were built on
randomly selected training samples of the same size as the personalized
models

We applied a form of predictive mean matching to determine the relative
weights for each matching variable using the following steps.28 First,
we imputed a 365- day TUG value for each patient in the training dataset
via the brokenstick package in R (because data were collected at
irregular timepoints).29 Next, we created a linear model to estimate the
imputed 365-day TUG value using our matching variables of interest We
then used this linear model to estimate the 365-day TUG value for (a)
each patient in the training dataset and (b) the index patient. Finally,
the patients from the training dataset with the closest predicted
365-day TUG value to the index patient were selected as matches; we used
35 patient matches based on our previous work.12

We modeled the observed TUG data from the matching patient records to
form the index patient's predicted TUG recovery trajectory. We used
Generalized Additive Models for Location, Scale, and Shape (GAMLSS) to
flexibly model the median, variance, and skewness of TUG recovery from
postoperative days 1-425.30 The GAMLSS model included a cubic spline
smoother with 3 degrees of freedom for the location parameter and 1
degree of freedom for the shape parameter.

A previous study suggested that in personalized medicine, using patient
similarity in data-driven analysis of patient cohorts will signifcantly
assist physicians to make informed decisions and choose the most
appropriate clinical trial

(why it is reasonable and useful)

\begin{itemize}
\item
  \textbf{what type of study}: The EPIC Observational Study is a
  prospective, multicenter, observational longitudinal study
\item
  \textbf{multiple center}: The data was collected through the Cystic
  Fibrosis Foundation Patient Registry (CFFPR)
\item
  \textbf{already in use}: The design of the EPIC study has been
  previously reported {[}28,29{]}. Including 59 centers
\item
  \textbf{exclusive informations}: Include demographic and risk factor
  data, respiratory cultures, clinical encounter information, and
  clinical outcomes, such as bacterial infections or pulmonary
  exacerbations
\item
  \textbf{what is the goal}:
\item
  \textbf{our goal}:
\end{itemize}

\hypertarget{subjects}{%
\subsubsection{Subjects}\label{subjects}}

\begin{itemize}
\item
  \textbf{check for time!!}: 1772 participants enrolled between
  \emph{2004 and 2006}. The data cut-off for the current analysis was
  \emph{December 2013}.
\item
  The enrolled children with a confirmed diagnosis of CF before the age
  of 12 without exposure of \emph{Pseudomonas aeruginosa}
\item
  Based on either never infected or culture negative for at least 2
  years); with respiratory cultures negative for \emph{Pseudomonas
  aeruginosa}
\item
  Either no prior isolation of Pseudomonas aeruginosa, or at least a
  two-year period with no isolation of P. aeruginosa (Treggiari et al.,
  2009).
\item
  The beginning of follow-up was defined as an individual's earliest
  registry entry or pulmonary exacerbation
\item
  a registry entry could have occurred before their enrollment in the
  EPIC Observational Study
\item
  End of follow-up in the study was defined as a patient's latest CF
  registry encounter or pulmonary exacerbation.
\item
  \textbf{sample size}: Of the \textbf{1772} children enrolled,
  \textbf{76497} visit observations in the EPIC OBS study
\item
  we identified \textbf{1325} individuals with usable data
\item
  1325 for the final data analysis
\item
  \textbf{how many} missing or lost to-follow-up with time-varying
  covariates and outcomes
\item
  \textbf{\%} had one survey record, \textbf{\%} had a second, and
  \textbf{\%} had a third, and the remaining \textbf{\%} had
  \textbf{\#th} surveys.
\item
  Total of \textbf{\# obs} in \textbf{\# subject} individuals were
  identified after eliminating registration age over 5
\item
  Total of \textbf{\# obs} in \textbf{\# subject} individuals were
  identified after eliminating follow up time less than 5 years
\item
  Total of \textbf{\# obs} in \textbf{\# subject} individuals were
  identified after eliminating avaiable visit times less than 10
\item
  \textbf{testing/training}: randomly
\end{itemize}

\hypertarget{table1-information}{%
\subsubsection{Table1 Information}\label{table1-information}}

\begin{itemize}
\item
  half of the cohort was female (49\%)
\item
  most were Caucasian (95\%)
\item
  Our genotype of interest was defined by the number of confirmed
  \(\Delta F508\) (\texttt{F508del}) mutations (\texttt{0}: without
  mutation, \texttt{1}: with one allele mutated, or \texttt{2}: with
  both alells mutated), where \texttt{0} included both those with both
  alleles missing or those with unknown status with respect to this
  mutation.
\item
  Homozygosity of \(\Delta F508\) (\(\Delta F508/\Delta F508\))
\item
  the most common CF-causing mutation (Cystic Fibrosis Foundation,

  \begin{enumerate}
  \def\labelenumi{\arabic{enumi})}
  \setcounter{enumi}{2015}
  \tightlist
  \item
    is associated with increased risk of infection and lower lung
    function in CF (Rosenfeld et al., 2012, Sanders et al., 2010).
  \end{enumerate}
\item
  the majority were F508del homozygous (\textbf{\%}),
\item
  the median weight percentile at enrollment was 36.7 \textbf{(IQR: )}.
\item
  The median age at enrollment was 5.8 \textbf{(IQR: )} years
\item
  median length of follow-up 7.8 \textbf{(IQR:)} years
\end{itemize}

\hypertarget{tools}{%
\subsubsection{Tools}\label{tools}}

the \textbf{brokenstick v 2.0.0} package. The package contains tools to
fit the broken stick model to data, export the fitted model's
parameters, create imputed values of the model, and predict broken stick
estimates for new data.

\hypertarget{brokenstick-model}{%
\subsection{brokenstick model}\label{brokenstick-model}}

Substantive researchers often favour repeated measures over the use of
linear mixed models because of their simplicity.

For example, we can easily fit a subject-level model to predict future
outcomes conditional on earlier data with repeated measures data.

While such simple regression models may be less efficient than modelling
the complete data (Diggle, Heagerty, Liang, and Zeger 2002, Sec. 6.1),

The broken stick model requires a specification of a sensible set of
time points at which the measurements ideally should have been taken.

\begin{itemize}
\item
  For each subject, the model predicts or imputes hypothetical
  observations at those times
\item
  so the substantive analysis applies to the repeated measures instead
  of the irregular data.
\end{itemize}

applications for brokenstick model:

\begin{itemize}
\tightlist
\item
  to approximate individual trajectories by a series of connected
  straight lines; • to align irregularly observed curves to a joint age
  grid;
\item
  to impute realisations of individual trajectories;
\item
  to estimate the time-to-time correlation matrix;
\item
  to predict future observations.
\end{itemize}

provide the examples:

\begin{itemize}
\item
  the points are close to the scheduled ages (indicated by vertical
  lines), especially in the first half-year.
\item
  bservation times vary more for older children.
\item
  several children have one or more missing visits.
\item
  some children had fairly close visits and vary periodically
\item
  special cases of dropped out after certain time
\end{itemize}

\hypertarget{people-like-me-methods}{%
\subsection{people-like-me methods}\label{people-like-me-methods}}

Assumptions:

\begin{itemize}
\item
  The trajectory between break ages follows a strict linear functino.
  This assumption may fail for processes that are convex or concave in
  time. For example, human height growth in centimeters growth is
  concave, so setting breakpoints far apart results introduces
  systematic model bias. Modeling height SDS instead of raw height will
  prevent this bias.
\item
  The broken stick estimates follow a joint multivariate normal
  distribution. As this assumption may fail for skewed measurements, it
  could be beneficial to transform the outcomes so that their
  distribution will be closer to normal.
\item
  The data are Missing at Random (MAR) given the outcomes from all
  subjects at all observation times. This assumption is restrictive in
  the sense that missingness may only depend on the observed outcomes,
  and not on covariates other than time.
\end{itemize}

At the same time, the assumption is liberal in the sense that the
missingness may depend on future outcomes. While this MAR-future
assumption is unusual in the literature on drop-out and observation time
models, it is a sensible strategy for creating imputations that preserve
relations over time, especially for intermittent missing data. Of
course, the subsequent substantive analysis on the imputed data needs to
be aware of the causal direction of time.

\hypertarget{kr-fucntion}{%
\subsubsection{KR fucntion???}\label{kr-fucntion}}

The brokenstick package provides another alternative, the Kasim-
Raudenbush (KR) sampler (Kasim and Raudenbush 1998). The method
simulates draws from the posterior distributions of parameters from a
two-level normal model with heterogeneous within-subject variances. The
speed of the Kasim-Raudenbush sampler is almost insensitive to the
number of random effects and depends primarily on the total number of
iterations and somewhat on sample size.

\hypertarget{discussion}{%
\section{Discussion}\label{discussion}}

\hypertarget{reference}{%
\section{Reference}\label{reference}}

\hypertarget{supplementary}{%
\section{Supplementary}\label{supplementary}}

\hypertarget{examples-of-citations-figures-tables-references}{%
\section{Examples of citations, figures, tables,
references}\label{examples-of-citations-figures-tables-references}}

\label{sec:others}

You can insert references. Here is some text (Kour and Saabne 2014b,
2014a) and see Hadash et al. (2018).

The documentation for \verb+natbib+ may be found at

You can use custom blocks with LaTeX support from \textbf{rmarkdown} to
create environment.

\begin{center}
\url{http://mirrors.ctan.org/macros/latex/contrib/natbib/natnotes.pdf\%7D}

\end{center}

Of note is the command \verb+\citet+, which produces citations
appropriate for use in inline text.

You can insert LaTeX environment directly too.

\begin{verbatim}
   \citet{hasselmo} investigated\dots
\end{verbatim}

produces

\begin{quote}
  Hasselmo, et al.\ (1995) investigated\dots
\end{quote}

\begin{center}
  \url{https://www.ctan.org/pkg/booktabs}
\end{center}

\hypertarget{figures}{%
\subsection{Figures}\label{figures}}

You can insert figure using LaTeX directly.

See Figure \ref{fig:fig1}. Here is how you add footnotes. {[}\^{}Sample
of the first footnote.{]}

\begin{figure}
  \centering
  \fbox{\rule[-.5cm]{4cm}{4cm} \rule[-.5cm]{4cm}{0cm}}
  \caption{Sample figure caption.}
  \label{fig:fig1}
\end{figure}

But you can also do that using R.

\begin{Shaded}
\begin{Highlighting}[]
\FunctionTok{plot}\NormalTok{(mtcars}\SpecialCharTok{$}\NormalTok{mpg)}
\end{Highlighting}
\end{Shaded}

\begin{figure}
\centering
\includegraphics{09_paper_files/figure-latex/fig2-1.pdf}
\caption{Another sample figure}
\end{figure}

You can use \textbf{bookdown} to allow references for Tables and
Figures.

\hypertarget{tables}{%
\subsection{Tables}\label{tables}}

Below we can see how to use tables.

See awesome Table\textasciitilde{}\ref{tab:table} which is written
directly in LaTeX in source Rmd file.

\begin{table}
 \caption{Sample table title}
  \centering
  \begin{tabular}{lll}
    \toprule
    \multicolumn{2}{c}{Part}                   \\
    \cmidrule(r){1-2}
    Name     & Description     & Size ($\mu$m) \\
    \midrule
    Dendrite & Input terminal  & $\sim$100     \\
    Axon     & Output terminal & $\sim$10      \\
    Soma     & Cell body       & up to $10^6$  \\
    \bottomrule
  \end{tabular}
  \label{tab:table}
\end{table}

You can also use R code for that.

\begin{Shaded}
\begin{Highlighting}[]
\NormalTok{knitr}\SpecialCharTok{::}\FunctionTok{kable}\NormalTok{(}\FunctionTok{head}\NormalTok{(mtcars), }\AttributeTok{caption =} \StringTok{"Head of mtcars table"}\NormalTok{)}
\end{Highlighting}
\end{Shaded}

\begin{longtable}[]{@{}
  >{\raggedright\arraybackslash}p{(\columnwidth - 22\tabcolsep) * \real{0.2609}}
  >{\raggedleft\arraybackslash}p{(\columnwidth - 22\tabcolsep) * \real{0.0725}}
  >{\raggedleft\arraybackslash}p{(\columnwidth - 22\tabcolsep) * \real{0.0580}}
  >{\raggedleft\arraybackslash}p{(\columnwidth - 22\tabcolsep) * \real{0.0725}}
  >{\raggedleft\arraybackslash}p{(\columnwidth - 22\tabcolsep) * \real{0.0580}}
  >{\raggedleft\arraybackslash}p{(\columnwidth - 22\tabcolsep) * \real{0.0725}}
  >{\raggedleft\arraybackslash}p{(\columnwidth - 22\tabcolsep) * \real{0.0870}}
  >{\raggedleft\arraybackslash}p{(\columnwidth - 22\tabcolsep) * \real{0.0870}}
  >{\raggedleft\arraybackslash}p{(\columnwidth - 22\tabcolsep) * \real{0.0435}}
  >{\raggedleft\arraybackslash}p{(\columnwidth - 22\tabcolsep) * \real{0.0435}}
  >{\raggedleft\arraybackslash}p{(\columnwidth - 22\tabcolsep) * \real{0.0725}}
  >{\raggedleft\arraybackslash}p{(\columnwidth - 22\tabcolsep) * \real{0.0725}}@{}}
\caption{Head of mtcars table}\tabularnewline
\toprule()
\begin{minipage}[b]{\linewidth}\raggedright
\end{minipage} & \begin{minipage}[b]{\linewidth}\raggedleft
mpg
\end{minipage} & \begin{minipage}[b]{\linewidth}\raggedleft
cyl
\end{minipage} & \begin{minipage}[b]{\linewidth}\raggedleft
disp
\end{minipage} & \begin{minipage}[b]{\linewidth}\raggedleft
hp
\end{minipage} & \begin{minipage}[b]{\linewidth}\raggedleft
drat
\end{minipage} & \begin{minipage}[b]{\linewidth}\raggedleft
wt
\end{minipage} & \begin{minipage}[b]{\linewidth}\raggedleft
qsec
\end{minipage} & \begin{minipage}[b]{\linewidth}\raggedleft
vs
\end{minipage} & \begin{minipage}[b]{\linewidth}\raggedleft
am
\end{minipage} & \begin{minipage}[b]{\linewidth}\raggedleft
gear
\end{minipage} & \begin{minipage}[b]{\linewidth}\raggedleft
carb
\end{minipage} \\
\midrule()
\endfirsthead
\toprule()
\begin{minipage}[b]{\linewidth}\raggedright
\end{minipage} & \begin{minipage}[b]{\linewidth}\raggedleft
mpg
\end{minipage} & \begin{minipage}[b]{\linewidth}\raggedleft
cyl
\end{minipage} & \begin{minipage}[b]{\linewidth}\raggedleft
disp
\end{minipage} & \begin{minipage}[b]{\linewidth}\raggedleft
hp
\end{minipage} & \begin{minipage}[b]{\linewidth}\raggedleft
drat
\end{minipage} & \begin{minipage}[b]{\linewidth}\raggedleft
wt
\end{minipage} & \begin{minipage}[b]{\linewidth}\raggedleft
qsec
\end{minipage} & \begin{minipage}[b]{\linewidth}\raggedleft
vs
\end{minipage} & \begin{minipage}[b]{\linewidth}\raggedleft
am
\end{minipage} & \begin{minipage}[b]{\linewidth}\raggedleft
gear
\end{minipage} & \begin{minipage}[b]{\linewidth}\raggedleft
carb
\end{minipage} \\
\midrule()
\endhead
Mazda RX4 & 21.0 & 6 & 160 & 110 & 3.90 & 2.620 & 16.46 & 0 & 1 & 4 &
4 \\
Mazda RX4 Wag & 21.0 & 6 & 160 & 110 & 3.90 & 2.875 & 17.02 & 0 & 1 & 4
& 4 \\
Datsun 710 & 22.8 & 4 & 108 & 93 & 3.85 & 2.320 & 18.61 & 1 & 1 & 4 &
1 \\
Hornet 4 Drive & 21.4 & 6 & 258 & 110 & 3.08 & 3.215 & 19.44 & 1 & 0 & 3
& 1 \\
Hornet Sportabout & 18.7 & 8 & 360 & 175 & 3.15 & 3.440 & 17.02 & 0 & 0
& 3 & 2 \\
Valiant & 18.1 & 6 & 225 & 105 & 2.76 & 3.460 & 20.22 & 1 & 0 & 3 & 1 \\
\bottomrule()
\end{longtable}

\hypertarget{lists}{%
\subsection{Lists}\label{lists}}

\begin{itemize}
\tightlist
\item
  Item 1
\item
  Item 2
\item
  Item 3
\end{itemize}

\hypertarget{refs}{}
\begin{CSLReferences}{1}{0}
\leavevmode\vadjust pre{\hypertarget{ref-hadash2018estimate}{}}%
Hadash, Guy, Einat Kermany, Boaz Carmeli, Ofer Lavi, George Kour, and
Alon Jacovi. 2018. {``Estimate and Replace: A Novel Approach to
Integrating Deep Neural Networks with Existing Applications.''}
\emph{arXiv Preprint arXiv:1804.09028}.

\leavevmode\vadjust pre{\hypertarget{ref-kour2014fast}{}}%
Kour, George, and Raid Saabne. 2014a. {``Fast Classification of
Handwritten on-Line Arabic Characters.''} In \emph{Soft Computing and
Pattern Recognition (SoCPaR), 2014 6th International Conference of},
312--18. IEEE.

\leavevmode\vadjust pre{\hypertarget{ref-kour2014real}{}}%
---------. 2014b. {``Real-Time Segmentation of on-Line Handwritten
Arabic Script.''} In \emph{Frontiers in Handwriting Recognition (ICFHR),
2014 14th International Conference on}, 417--22. IEEE.

\end{CSLReferences}

\bibliographystyle{unsrt}
\bibliography{references.bib}


\end{document}
